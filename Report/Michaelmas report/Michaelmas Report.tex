\documentclass[a4paper,11pt]{article}
\usepackage{color}
\usepackage{graphicx}
\usepackage{subcaption}
\usepackage{wrapfig}
\usepackage[export]{adjustbox}
\usepackage{geometry}
\usepackage{setspace}
\doublespacing
\geometry{legalpaper, portrait, margin=2cm}
\begin{document}
\title{Open Science Hardware Setup for investigating the Stability of Organic Solar Cells - Interim Report}
\author{Samuel Mendis}
\date{\today}
\maketitle
\pagebreak
\section{Introduction}
Solar cells are becoming an increasingly important factor in the fight against climate change. Silicon solar cells are the industry standard being used worldwide in a multitude of applications. Organic solar cells are showing potential to become an integral part for the  This report will outline the current development of the open hardware setup being designed to investigate the stability of organic solar cells. This project 
\section{Where the project is?}
Currently the project is at the manufacturing stage. A full design has been completed for the testing container pictured in (figure 1) which as completed using the open source software OpensCAD. As mentioned in the introduction, being an open hardware project is one of the key aspects, thereby forcing the choice of an open-source CAD software. OpensCAD was chosen as it satisfied this parameter as well the bonus of having extensive documentation allowing the designs to be completed with relative ease. Once the software was chosen, the next steps were to decide on the size and shape of the container. One of the Advanced Functional Materials Department (AFMD) researchers named () had already developed some type of testing container which is where inspiration was drawn from. 
\\
\\
Initially a square box was chosen with very little room inside for the implementation of devices other than the substrate. The substrate used carries 8 solar cells with dimensions of 30 mm * 30 mm and a thickness of 1 mm, this can be seen in figure (2) which is a model of the substrate used provided by Dr. Grey... To cary the substrate in the container a substrate holder was designed to fit within the outer shell of the container allowing a more modular design for the container. 




\end{document}